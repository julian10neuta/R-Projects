% Options for packages loaded elsewhere
\PassOptionsToPackage{unicode}{hyperref}
\PassOptionsToPackage{hyphens}{url}
%
\documentclass[
]{article}
\usepackage{amsmath,amssymb}
\usepackage{iftex}
\ifPDFTeX
  \usepackage[T1]{fontenc}
  \usepackage[utf8]{inputenc}
  \usepackage{textcomp} % provide euro and other symbols
\else % if luatex or xetex
  \usepackage{unicode-math} % this also loads fontspec
  \defaultfontfeatures{Scale=MatchLowercase}
  \defaultfontfeatures[\rmfamily]{Ligatures=TeX,Scale=1}
\fi
\usepackage{lmodern}
\ifPDFTeX\else
  % xetex/luatex font selection
\fi
% Use upquote if available, for straight quotes in verbatim environments
\IfFileExists{upquote.sty}{\usepackage{upquote}}{}
\IfFileExists{microtype.sty}{% use microtype if available
  \usepackage[]{microtype}
  \UseMicrotypeSet[protrusion]{basicmath} % disable protrusion for tt fonts
}{}
\makeatletter
\@ifundefined{KOMAClassName}{% if non-KOMA class
  \IfFileExists{parskip.sty}{%
    \usepackage{parskip}
  }{% else
    \setlength{\parindent}{0pt}
    \setlength{\parskip}{6pt plus 2pt minus 1pt}}
}{% if KOMA class
  \KOMAoptions{parskip=half}}
\makeatother
\usepackage{xcolor}
\usepackage[margin=1in]{geometry}
\usepackage{graphicx}
\makeatletter
\def\maxwidth{\ifdim\Gin@nat@width>\linewidth\linewidth\else\Gin@nat@width\fi}
\def\maxheight{\ifdim\Gin@nat@height>\textheight\textheight\else\Gin@nat@height\fi}
\makeatother
% Scale images if necessary, so that they will not overflow the page
% margins by default, and it is still possible to overwrite the defaults
% using explicit options in \includegraphics[width, height, ...]{}
\setkeys{Gin}{width=\maxwidth,height=\maxheight,keepaspectratio}
% Set default figure placement to htbp
\makeatletter
\def\fps@figure{htbp}
\makeatother
\setlength{\emergencystretch}{3em} % prevent overfull lines
\providecommand{\tightlist}{%
  \setlength{\itemsep}{0pt}\setlength{\parskip}{0pt}}
\setcounter{secnumdepth}{-\maxdimen} % remove section numbering
\ifLuaTeX
  \usepackage{selnolig}  % disable illegal ligatures
\fi
\usepackage{bookmark}
\IfFileExists{xurl.sty}{\usepackage{xurl}}{} % add URL line breaks if available
\urlstyle{same}
\hypersetup{
  pdftitle={Parcial Distribuciones},
  pdfauthor={Julian Velandia},
  hidelinks,
  pdfcreator={LaTeX via pandoc}}

\title{Parcial Distribuciones}
\author{Julian Velandia}
\date{2025-07-08}

\begin{document}
\maketitle

\subsection{Primer punto}\label{primer-punto}

\begin{enumerate}
\def\labelenumi{\arabic{enumi}.}
\tightlist
\item
  Un examen de opción múltiple tiene 15 preguntas, cada una con cinco
  posibles respuestas, sólo una de las cuales es correcta. Suponga que
  uno de los estudiantes que hace el examen contesta cada una de las
  preguntas con una adivinación aleatoria independiente. ¿Cuál es la
  probabilidad de que conteste correctamente al menos 6 preguntas?
\end{enumerate}

\begin{verbatim}
## la probabilidad de que conteste correctamente al menos 6 preguntas P(X>=6) =  0.06105143
\end{verbatim}

\begin{center}\includegraphics{PARCIAL-3-DISTRIBUCIONES_files/figure-latex/unnamed-chunk-1-1} \end{center}

\subsection{Segundo punto}\label{segundo-punto}

Un aparato detector de incendios utiliza tres celdas sensibles a la
temperatura que actúan de manera independiente una de otra, en forma tal
que una o más pueden activar la alarma. Cada celda presenta una
probabilidad de p = 0.7 de activar la alarma cuando la temperatura
alcanza 100°C o más. Sea X igual al número de celdas que activan la
alarma cuando la temperatura alcanza los 100°. Encuentre la distribución
de probabilidad para X y con base en ella, encuentre la probabilidad de
que la alarma funcione cuando la temperatura alcance los 100°.

\begin{verbatim}
## la probabilidad de que la alarma funcione cuando la temperatura alcance los 100° P(X > 0) =  0.973
\end{verbatim}

\begin{center}\includegraphics{PARCIAL-3-DISTRIBUCIONES_files/figure-latex/unnamed-chunk-2-1} \end{center}

\subsection{Tercer punto}\label{tercer-punto}

Un almacén contiene diez máquinas impresoras, cuatro de las cuales son
defectuosas. Una compañía selecciona cinco de las máquinas al azar
pensando que todas están en buenas condiciones. ¿Cuál es la probabilidad
de que las cinco no sean defectuosas?

\begin{verbatim}
## La probabilidad de que las cinco no sean defectuosas es: 0.02380952
\end{verbatim}

\begin{center}\includegraphics{PARCIAL-3-DISTRIBUCIONES_files/figure-latex/unnamed-chunk-3-1} \end{center}

\subsection{Cuarto punto}\label{cuarto-punto}

Para el ejercicio 3, la compañía repara las impresoras defectuosas a un
costo de \$50 cada una. Encuentre la media y la varianza del costo total
de reparación. (Redondee al entero más cercano)

\begin{verbatim}
## Costo medio de reparación: 100
\end{verbatim}

\begin{verbatim}
## Varianza del costo total de reparación: 1667
\end{verbatim}

\subsection{Quinto punto}\label{quinto-punto}

\begin{enumerate}
\def\labelenumi{\arabic{enumi}.}
\setcounter{enumi}{4}
\tightlist
\item
  Llegan clientes a un mostrador de salida en una tienda de
  departamentos de acuerdo con una distribución de Poisson, a un
  promedio de siete por hora. Durante una hora determinada, ¿Cuál es la
  probabilidad de que no lleguen más de tres clientes?
\end{enumerate}

\begin{verbatim}
## La probabilidad de que no lleguen más de tres clientes es: 0.08176542
\end{verbatim}

\begin{center}\includegraphics{PARCIAL-3-DISTRIBUCIONES_files/figure-latex/unnamed-chunk-5-1} \end{center}

\subsection{Sexto punto}\label{sexto-punto}

\begin{enumerate}
\def\labelenumi{\arabic{enumi}.}
\setcounter{enumi}{5}
\tightlist
\item
  Al estudiar bajas cotizaciones para contratos de embarques, una
  empresa fabricante de microcomputadoras encuentra que los contratos
  interestatales tienen bajas cotizaciones que están uniformemente
  distribuidas entre 20 y 25, en unidades de miles de dólares. Encuentre
  la probabilidad de que la baja cotización en el siguiente contrato
  interestatal sea de más de \$24,000.
\end{enumerate}

\begin{verbatim}
## La probabilidad de que la baja cotización en el siguiente contrato interestatal sea de más de $24,000 es: 0.2
\end{verbatim}

\begin{center}\includegraphics{PARCIAL-3-DISTRIBUCIONES_files/figure-latex/unnamed-chunk-6-1} \end{center}

\subsection{Septimo punto}\label{septimo-punto}

\begin{enumerate}
\def\labelenumi{\arabic{enumi}.}
\setcounter{enumi}{6}
\tightlist
\item
  Se supone que las calificaciones de un examen están normalmente
  distribuidas con media de 78 y varianza de 36. Suponga que los
  estudiantes que alcancen el 10\% más alto de esta distribución reciben
  una calificación de A. ¿Cuál es la calificación mínima que un
  estudiante debe recibir para ganar una calificación de A?
\end{enumerate}

\begin{verbatim}
## [1] 36
\end{verbatim}

\begin{verbatim}
## La nota mínima para una A es: 85.68931
\end{verbatim}

\begin{center}\includegraphics{PARCIAL-3-DISTRIBUCIONES_files/figure-latex/unnamed-chunk-7-1} \end{center}

\subsection{Octavo punto}\label{octavo-punto}

\begin{enumerate}
\def\labelenumi{\arabic{enumi}.}
\setcounter{enumi}{7}
\tightlist
\item
  Los exámenes de admisión de la Unal se aplican a miles de estudiantes
  cada semestre. Las secciones de matemáticas de cada uno de estos
  exámenes producen calificaciones que están normalmente distribuidas,
  en forma aproximada. En años recientes las calificaciones de exámenes
  de matemáticas han promediado 480 con desviación estándar de 100. La
  facultad de ingeniería establece 550 como calificación mínima en
  matemáticas para estudiantes de nuevo ingreso. ¿Qué porcentaje de
  estudiantes obtendrá una calificación por debajo de 550 en el próximo
  semestre?
\end{enumerate}

\begin{verbatim}
## La probabilidad  0.7580363
\end{verbatim}

\begin{center}\includegraphics{PARCIAL-3-DISTRIBUCIONES_files/figure-latex/unnamed-chunk-8-1} \end{center}

\subsection{Noveno punto}\label{noveno-punto}

\begin{enumerate}
\def\labelenumi{\arabic{enumi}.}
\setcounter{enumi}{8}
\tightlist
\item
  Una planta de producción industrial registra que el tiempo (en días)
  que transcurre entre una falla y la siguiente en una máquina
  específica sigue una distribución exponencial con parámetro λ=0,2.
  ¿Cuál es la probabilidad de que la máquina funcione al menos 10 días
  sin fallar?
\end{enumerate}

\begin{verbatim}
## La probabilidad de que funcione al menos 10 días sin fallar es: 0.1353353
\end{verbatim}

\end{document}
